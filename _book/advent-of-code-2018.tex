\documentclass[]{book}
\usepackage{lmodern}
\usepackage{amssymb,amsmath}
\usepackage{ifxetex,ifluatex}
\usepackage{fixltx2e} % provides \textsubscript
\ifnum 0\ifxetex 1\fi\ifluatex 1\fi=0 % if pdftex
  \usepackage[T1]{fontenc}
  \usepackage[utf8]{inputenc}
\else % if luatex or xelatex
  \ifxetex
    \usepackage{mathspec}
  \else
    \usepackage{fontspec}
  \fi
  \defaultfontfeatures{Ligatures=TeX,Scale=MatchLowercase}
\fi
% use upquote if available, for straight quotes in verbatim environments
\IfFileExists{upquote.sty}{\usepackage{upquote}}{}
% use microtype if available
\IfFileExists{microtype.sty}{%
\usepackage{microtype}
\UseMicrotypeSet[protrusion]{basicmath} % disable protrusion for tt fonts
}{}
\usepackage[margin=1in]{geometry}
\usepackage{hyperref}
\hypersetup{unicode=true,
            pdftitle={Advent of Code 2018},
            pdfauthor={Maria Paula Caldas},
            pdfborder={0 0 0},
            breaklinks=true}
\urlstyle{same}  % don't use monospace font for urls
\usepackage{natbib}
\bibliographystyle{apalike}
\usepackage{color}
\usepackage{fancyvrb}
\newcommand{\VerbBar}{|}
\newcommand{\VERB}{\Verb[commandchars=\\\{\}]}
\DefineVerbatimEnvironment{Highlighting}{Verbatim}{commandchars=\\\{\}}
% Add ',fontsize=\small' for more characters per line
\usepackage{framed}
\definecolor{shadecolor}{RGB}{248,248,248}
\newenvironment{Shaded}{\begin{snugshade}}{\end{snugshade}}
\newcommand{\KeywordTok}[1]{\textcolor[rgb]{0.13,0.29,0.53}{\textbf{#1}}}
\newcommand{\DataTypeTok}[1]{\textcolor[rgb]{0.13,0.29,0.53}{#1}}
\newcommand{\DecValTok}[1]{\textcolor[rgb]{0.00,0.00,0.81}{#1}}
\newcommand{\BaseNTok}[1]{\textcolor[rgb]{0.00,0.00,0.81}{#1}}
\newcommand{\FloatTok}[1]{\textcolor[rgb]{0.00,0.00,0.81}{#1}}
\newcommand{\ConstantTok}[1]{\textcolor[rgb]{0.00,0.00,0.00}{#1}}
\newcommand{\CharTok}[1]{\textcolor[rgb]{0.31,0.60,0.02}{#1}}
\newcommand{\SpecialCharTok}[1]{\textcolor[rgb]{0.00,0.00,0.00}{#1}}
\newcommand{\StringTok}[1]{\textcolor[rgb]{0.31,0.60,0.02}{#1}}
\newcommand{\VerbatimStringTok}[1]{\textcolor[rgb]{0.31,0.60,0.02}{#1}}
\newcommand{\SpecialStringTok}[1]{\textcolor[rgb]{0.31,0.60,0.02}{#1}}
\newcommand{\ImportTok}[1]{#1}
\newcommand{\CommentTok}[1]{\textcolor[rgb]{0.56,0.35,0.01}{\textit{#1}}}
\newcommand{\DocumentationTok}[1]{\textcolor[rgb]{0.56,0.35,0.01}{\textbf{\textit{#1}}}}
\newcommand{\AnnotationTok}[1]{\textcolor[rgb]{0.56,0.35,0.01}{\textbf{\textit{#1}}}}
\newcommand{\CommentVarTok}[1]{\textcolor[rgb]{0.56,0.35,0.01}{\textbf{\textit{#1}}}}
\newcommand{\OtherTok}[1]{\textcolor[rgb]{0.56,0.35,0.01}{#1}}
\newcommand{\FunctionTok}[1]{\textcolor[rgb]{0.00,0.00,0.00}{#1}}
\newcommand{\VariableTok}[1]{\textcolor[rgb]{0.00,0.00,0.00}{#1}}
\newcommand{\ControlFlowTok}[1]{\textcolor[rgb]{0.13,0.29,0.53}{\textbf{#1}}}
\newcommand{\OperatorTok}[1]{\textcolor[rgb]{0.81,0.36,0.00}{\textbf{#1}}}
\newcommand{\BuiltInTok}[1]{#1}
\newcommand{\ExtensionTok}[1]{#1}
\newcommand{\PreprocessorTok}[1]{\textcolor[rgb]{0.56,0.35,0.01}{\textit{#1}}}
\newcommand{\AttributeTok}[1]{\textcolor[rgb]{0.77,0.63,0.00}{#1}}
\newcommand{\RegionMarkerTok}[1]{#1}
\newcommand{\InformationTok}[1]{\textcolor[rgb]{0.56,0.35,0.01}{\textbf{\textit{#1}}}}
\newcommand{\WarningTok}[1]{\textcolor[rgb]{0.56,0.35,0.01}{\textbf{\textit{#1}}}}
\newcommand{\AlertTok}[1]{\textcolor[rgb]{0.94,0.16,0.16}{#1}}
\newcommand{\ErrorTok}[1]{\textcolor[rgb]{0.64,0.00,0.00}{\textbf{#1}}}
\newcommand{\NormalTok}[1]{#1}
\usepackage{longtable,booktabs}
\usepackage{graphicx,grffile}
\makeatletter
\def\maxwidth{\ifdim\Gin@nat@width>\linewidth\linewidth\else\Gin@nat@width\fi}
\def\maxheight{\ifdim\Gin@nat@height>\textheight\textheight\else\Gin@nat@height\fi}
\makeatother
% Scale images if necessary, so that they will not overflow the page
% margins by default, and it is still possible to overwrite the defaults
% using explicit options in \includegraphics[width, height, ...]{}
\setkeys{Gin}{width=\maxwidth,height=\maxheight,keepaspectratio}
\IfFileExists{parskip.sty}{%
\usepackage{parskip}
}{% else
\setlength{\parindent}{0pt}
\setlength{\parskip}{6pt plus 2pt minus 1pt}
}
\setlength{\emergencystretch}{3em}  % prevent overfull lines
\providecommand{\tightlist}{%
  \setlength{\itemsep}{0pt}\setlength{\parskip}{0pt}}
\setcounter{secnumdepth}{5}
% Redefines (sub)paragraphs to behave more like sections
\ifx\paragraph\undefined\else
\let\oldparagraph\paragraph
\renewcommand{\paragraph}[1]{\oldparagraph{#1}\mbox{}}
\fi
\ifx\subparagraph\undefined\else
\let\oldsubparagraph\subparagraph
\renewcommand{\subparagraph}[1]{\oldsubparagraph{#1}\mbox{}}
\fi

%%% Use protect on footnotes to avoid problems with footnotes in titles
\let\rmarkdownfootnote\footnote%
\def\footnote{\protect\rmarkdownfootnote}

%%% Change title format to be more compact
\usepackage{titling}

% Create subtitle command for use in maketitle
\newcommand{\subtitle}[1]{
  \posttitle{
    \begin{center}\large#1\end{center}
    }
}

\setlength{\droptitle}{-2em}
  \title{Advent of Code 2018}
  \pretitle{\vspace{\droptitle}\centering\huge}
  \posttitle{\par}
  \author{Maria Paula Caldas}
  \preauthor{\centering\large\emph}
  \postauthor{\par}
  \predate{\centering\large\emph}
  \postdate{\par}
  \date{2018-12-02}

\usepackage{booktabs}

\usepackage{amsthm}
\newtheorem{theorem}{Theorem}[chapter]
\newtheorem{lemma}{Lemma}[chapter]
\theoremstyle{definition}
\newtheorem{definition}{Definition}[chapter]
\newtheorem{corollary}{Corollary}[chapter]
\newtheorem{proposition}{Proposition}[chapter]
\theoremstyle{definition}
\newtheorem{example}{Example}[chapter]
\theoremstyle{definition}
\newtheorem{exercise}{Exercise}[chapter]
\theoremstyle{remark}
\newtheorem*{remark}{Remark}
\newtheorem*{solution}{Solution}
\begin{document}
\maketitle

{
\setcounter{tocdepth}{1}
\tableofcontents
}
\chapter*{Session info}\label{session-info}
\addcontentsline{toc}{chapter}{Session info}

I will use the following packages:

\begin{Shaded}
\begin{Highlighting}[]
\KeywordTok{library}\NormalTok{(tidyverse)}
\end{Highlighting}
\end{Shaded}

My setup at the time:

\begin{verbatim}
## Session info -------------------------------------------------------------
\end{verbatim}

\begin{verbatim}
##  setting  value                       
##  version  R version 3.5.0 (2018-04-23)
##  system   x86_64, darwin15.6.0        
##  ui       X11                         
##  language (EN)                        
##  collate  en_US.UTF-8                 
##  tz       Europe/Paris                
##  date     2018-12-02
\end{verbatim}

\begin{verbatim}
## Packages -----------------------------------------------------------------
\end{verbatim}

\begin{verbatim}
##  package    * version date       source        
##  assertthat   0.2.0   2017-04-11 CRAN (R 3.5.0)
##  backports    1.1.2   2017-12-13 CRAN (R 3.5.0)
##  base       * 3.5.0   2018-04-24 local         
##  bindr        0.1.1   2018-03-13 CRAN (R 3.5.0)
##  bindrcpp   * 0.2.2   2018-03-29 CRAN (R 3.5.0)
##  bookdown     0.7     2018-02-18 CRAN (R 3.5.0)
##  broom        0.5.0   2018-07-17 CRAN (R 3.5.0)
##  cellranger   1.1.0   2016-07-27 CRAN (R 3.5.0)
##  cli          1.0.0   2017-11-05 CRAN (R 3.5.0)
##  colorspace   1.3-2   2016-12-14 CRAN (R 3.5.0)
##  compiler     3.5.0   2018-04-24 local         
##  crayon       1.3.4   2017-09-16 CRAN (R 3.5.0)
##  datasets   * 3.5.0   2018-04-24 local         
##  devtools     1.13.5  2018-02-18 CRAN (R 3.5.0)
##  digest       0.6.16  2018-08-22 cran (@0.6.16)
##  dplyr      * 0.7.6   2018-06-29 CRAN (R 3.5.1)
##  evaluate     0.10.1  2017-06-24 CRAN (R 3.5.0)
##  forcats    * 0.3.0   2018-02-19 CRAN (R 3.5.0)
##  ggplot2    * 3.0.0   2018-07-03 CRAN (R 3.5.0)
##  glue         1.3.0   2018-07-17 cran (@1.3.0) 
##  graphics   * 3.5.0   2018-04-24 local         
##  grDevices  * 3.5.0   2018-04-24 local         
##  grid         3.5.0   2018-04-24 local         
##  gtable       0.2.0   2016-02-26 CRAN (R 3.5.0)
##  haven        1.1.2   2018-06-27 CRAN (R 3.5.0)
##  hms          0.4.2   2018-03-10 CRAN (R 3.5.0)
##  htmltools    0.3.6   2017-04-28 CRAN (R 3.5.0)
##  httr         1.3.1   2017-08-20 CRAN (R 3.5.0)
##  jsonlite     1.5     2017-06-01 CRAN (R 3.5.0)
##  knitr        1.20    2018-02-20 CRAN (R 3.5.0)
##  lattice      0.20-35 2017-03-25 CRAN (R 3.5.0)
##  lazyeval     0.2.1   2017-10-29 CRAN (R 3.5.0)
##  lubridate    1.7.4   2018-04-11 CRAN (R 3.5.0)
##  magrittr     1.5     2014-11-22 CRAN (R 3.5.0)
##  memoise      1.1.0   2017-04-21 CRAN (R 3.5.0)
##  methods    * 3.5.0   2018-04-24 local         
##  modelr       0.1.2   2018-05-11 CRAN (R 3.5.0)
##  munsell      0.4.3   2016-02-13 CRAN (R 3.5.0)
##  nlme         3.1-137 2018-04-07 CRAN (R 3.5.0)
##  pillar       1.2.2   2018-04-26 CRAN (R 3.5.0)
##  pkgconfig    2.0.1   2017-03-21 CRAN (R 3.5.0)
##  plyr         1.8.4   2016-06-08 CRAN (R 3.5.0)
##  purrr      * 0.2.5   2018-05-29 CRAN (R 3.5.0)
##  R6           2.2.2   2017-06-17 CRAN (R 3.5.0)
##  Rcpp         0.12.16 2018-03-13 CRAN (R 3.5.0)
##  readr      * 1.1.1   2017-05-16 CRAN (R 3.5.0)
##  readxl       1.1.0   2018-04-20 CRAN (R 3.5.0)
##  rlang        0.2.2   2018-08-16 CRAN (R 3.5.0)
##  rmarkdown    1.9     2018-03-01 CRAN (R 3.5.0)
##  rprojroot    1.3-2   2018-01-03 CRAN (R 3.5.0)
##  rstudioapi   0.7     2017-09-07 CRAN (R 3.5.0)
##  rvest        0.3.2   2016-06-17 CRAN (R 3.5.0)
##  scales       0.5.0   2017-08-24 CRAN (R 3.5.0)
##  stats      * 3.5.0   2018-04-24 local         
##  stringi      1.1.7   2018-03-12 CRAN (R 3.5.0)
##  stringr    * 1.3.1   2018-05-10 CRAN (R 3.5.0)
##  tibble     * 1.4.2   2018-01-22 CRAN (R 3.5.0)
##  tidyr      * 0.8.1   2018-05-18 CRAN (R 3.5.0)
##  tidyselect   0.2.4   2018-02-26 CRAN (R 3.5.0)
##  tidyverse  * 1.2.1   2017-11-14 CRAN (R 3.5.0)
##  tools        3.5.0   2018-04-24 local         
##  utf8         1.1.3   2018-01-03 CRAN (R 3.5.0)
##  utils      * 3.5.0   2018-04-24 local         
##  withr        2.1.2   2018-03-15 CRAN (R 3.5.0)
##  xfun         0.1     2018-01-22 CRAN (R 3.5.0)
##  xml2         1.2.0   2018-01-24 CRAN (R 3.5.0)
##  yaml         2.2.0   2018-07-25 cran (@2.2.0)
\end{verbatim}

\chapter{Chronal Calibration}\label{day1}

Import puzzle imput for the day:

\begin{Shaded}
\begin{Highlighting}[]
\NormalTok{puzzle_input <-}\StringTok{ }\KeywordTok{as.numeric}\NormalTok{(}\KeywordTok{readLines}\NormalTok{(}\StringTok{"data-raw/day1.txt"}\NormalTok{, }\DataTypeTok{warn =} \OtherTok{FALSE}\NormalTok{))}
\end{Highlighting}
\end{Shaded}

\section{Puzzle 1}\label{puzzle-1}

\begin{quote}
Starting with a frequency of zero, what is the resulting frequency after
all of the changes in frequency have been applied?
\end{quote}

Easy enough:

\begin{Shaded}
\begin{Highlighting}[]
\KeywordTok{sum}\NormalTok{(puzzle_input)}
\end{Highlighting}
\end{Shaded}

\begin{verbatim}
## [1] 472
\end{verbatim}

\section{Puzzle 2}\label{puzzle-2}

\begin{quote}
What is the first frequency your device reaches twice?
\end{quote}

Let's create a cute little tibble.

\begin{Shaded}
\begin{Highlighting}[]
\NormalTok{(tib <-}\StringTok{ }\KeywordTok{tibble}\NormalTok{(}
  \DataTypeTok{input =}\NormalTok{ puzzle_input, }
  \DataTypeTok{cumsum =} \KeywordTok{cumsum}\NormalTok{(input),}
  \DataTypeTok{index =} \KeywordTok{seq}\NormalTok{(}\DecValTok{1}\OperatorTok{:}\KeywordTok{length}\NormalTok{(input))}
\NormalTok{  ))}
\end{Highlighting}
\end{Shaded}

\begin{verbatim}
## # A tibble: 1,000 x 3
##    input cumsum index
##    <dbl>  <dbl> <int>
##  1   -16    -16     1
##  2    12     -4     2
##  3   -18    -22     3
##  4    -1    -23     4
##  5     5    -18     5
##  6    -8    -26     6
##  7     9    -17     7
##  8   -15    -32     8
##  9    12    -20     9
## 10     6    -14    10
## # ... with 990 more rows
\end{verbatim}

First, let see how many frequencies have been reached more than once
(i.e.~have duplicates).

\begin{Shaded}
\begin{Highlighting}[]
\KeywordTok{count}\NormalTok{(tib, cumsum, }\DataTypeTok{sort =} \OtherTok{TRUE}\NormalTok{)}
\end{Highlighting}
\end{Shaded}

\begin{verbatim}
## # A tibble: 1,000 x 2
##    cumsum     n
##     <dbl> <int>
##  1   -111     1
##  2   -107     1
##  3   -103     1
##  4   -100     1
##  5    -98     1
##  6    -97     1
##  7    -96     1
##  8    -95     1
##  9    -94     1
## 10    -91     1
## # ... with 990 more rows
\end{verbatim}

Apparently none\ldots{} Maybe I should do it twice?

\begin{Shaded}
\begin{Highlighting}[]
\NormalTok{tib2 <-}\StringTok{ }\KeywordTok{tibble}\NormalTok{(}
  \DataTypeTok{input =} \KeywordTok{rep}\NormalTok{(puzzle_input, }\DecValTok{2}\NormalTok{), }
  \DataTypeTok{cumsum =} \KeywordTok{cumsum}\NormalTok{(input),}
  \DataTypeTok{index =} \KeywordTok{seq}\NormalTok{(}\DecValTok{1}\OperatorTok{:}\KeywordTok{length}\NormalTok{(input))}
\NormalTok{  )}

\NormalTok{tib2 }\OperatorTok\StringTok{ }
\StringTok{  }\KeywordTok{count}\NormalTok{(cumsum) }\OperatorTok\StringTok{ }
\StringTok{  }\KeywordTok{count}\NormalTok{(n)}
\end{Highlighting}
\end{Shaded}

\begin{verbatim}
## # A tibble: 1 x 2
##       n    nn
##   <int> <int>
## 1     1  2000
\end{verbatim}

So no, need to do it more than twice.

OK, let's just keep making the vector bigger until \textbf{at least} one
frequency is repeated. Also, let's just go back to base R.

\begin{Shaded}
\begin{Highlighting}[]
\NormalTok{growing_vector <-}\StringTok{ }\NormalTok{puzzle_input}

\ControlFlowTok{while}\NormalTok{ ( }\OperatorTok{!}\KeywordTok{any}\NormalTok{(}\KeywordTok{duplicated}\NormalTok{(}\KeywordTok{cumsum}\NormalTok{(growing_vector))) ) }
\NormalTok{  growing_vector <-}\StringTok{ }\KeywordTok{c}\NormalTok{(growing_vector, growing_vector)}
\end{Highlighting}
\end{Shaded}

The new vector is 256 times the size of the original input vector.

Now, let's get the frequencies:

\begin{Shaded}
\begin{Highlighting}[]
\NormalTok{cumsum_big_vector <-}\StringTok{ }\KeywordTok{cumsum}\NormalTok{(growing_vector)}
\end{Highlighting}
\end{Shaded}

And the indices of those that are repeated:

\begin{Shaded}
\begin{Highlighting}[]
\NormalTok{indx <-}\StringTok{ }\KeywordTok{duplicated}\NormalTok{(cumsum_big_vector)}
\end{Highlighting}
\end{Shaded}

Which allows me to get the \textbf{first} frequency that is repeated
twice:

\begin{Shaded}
\begin{Highlighting}[]
\NormalTok{cumsum_big_vector[indx][}\DecValTok{1}\NormalTok{]}
\end{Highlighting}
\end{Shaded}

\begin{verbatim}
## [1] 66932
\end{verbatim}

\subsection{Wrong attempts}\label{wrong-attempts}

At first, I thought the correct answer was:

\begin{Shaded}
\begin{Highlighting}[]
\NormalTok{growing_vector[indx][}\DecValTok{1}\NormalTok{]}
\end{Highlighting}
\end{Shaded}

\begin{verbatim}
## [1] 18
\end{verbatim}

\ldots{} which is in fact the change in frequency that leads to the
first frequency that appears twice!

\chapter{Inventory Management System}\label{day2}

Import puzzle imput for the day:

\begin{Shaded}
\begin{Highlighting}[]
\NormalTok{puzzle_input <-}\StringTok{ }\KeywordTok{readLines}\NormalTok{(}\StringTok{"data-raw/day2.txt"}\NormalTok{, }\DataTypeTok{warn =} \OtherTok{FALSE}\NormalTok{)}
\end{Highlighting}
\end{Shaded}

\section{Puzzle 1}\label{puzzle-1-1}

\begin{quote}
What is the checksum for your list of box IDs?
\end{quote}

\begin{Shaded}
\begin{Highlighting}[]
\NormalTok{any_rep <-}\StringTok{ }\ControlFlowTok{function}\NormalTok{(id, }\DataTypeTok{rep =} \KeywordTok{c}\NormalTok{(}\DecValTok{2}\NormalTok{, }\DecValTok{3}\NormalTok{)) \{}
\NormalTok{  count_per_letter <-}\StringTok{ }\KeywordTok{map_int}\NormalTok{(letters, }\OperatorTok{~}\StringTok{ }\KeywordTok{str_count}\NormalTok{(id, .x))}
  \KeywordTok{any}\NormalTok{(count_per_letter }\OperatorTok{==}\StringTok{ }\NormalTok{rep)}
  
\NormalTok{\}}

\KeywordTok{tibble}\NormalTok{(}
  \DataTypeTok{input =}\NormalTok{ puzzle_input, }
  \DataTypeTok{any_twice =} \KeywordTok{map_lgl}\NormalTok{(input, any_rep, }\DataTypeTok{rep =} \DecValTok{2}\NormalTok{),}
  \DataTypeTok{any_thrice =} \KeywordTok{map_lgl}\NormalTok{(input, any_rep, }\DataTypeTok{rep =} \DecValTok{3}\NormalTok{)}
\NormalTok{  ) }\OperatorTok\StringTok{ }
\StringTok{  }\KeywordTok{summarise}\NormalTok{(}\DataTypeTok{n_twice =} \KeywordTok{sum}\NormalTok{(any_twice), }\DataTypeTok{n_thrice =} \KeywordTok{sum}\NormalTok{(any_thrice)) }\OperatorTok\StringTok{ }
\StringTok{  }\KeywordTok{mutate}\NormalTok{(}\DataTypeTok{cumcheck =}\NormalTok{ n_twice }\OperatorTok{*}\StringTok{ }\NormalTok{n_thrice)}
\end{Highlighting}
\end{Shaded}

\begin{verbatim}
## # A tibble: 1 x 3
##   n_twice n_thrice cumcheck
##     <int>    <int>    <int>
## 1     247       31     7657
\end{verbatim}

\section{Puzzle 2}\label{puzzle-2-1}

\begin{quote}
What letters are common between the two correct box IDs? (In the example
above, this is found by removing the differing character from either ID,
producing \texttt{fgij}.)
\end{quote}

This one took me a while, but I learned a lot:

\begin{itemize}
\item
  Never forget to vectorise functions, especially those that are going
  to go through a \texttt{dplyr::mutate()}
\item
  \texttt{purrr::cross\_df()} is awesome, although not the right tool
  for this type of problem (I end up with twice the amount of
  combinations that I need)
\end{itemize}

\begin{Shaded}
\begin{Highlighting}[]
\NormalTok{are_almost_same <-}\StringTok{ }\ControlFlowTok{function}\NormalTok{(vector1, vector2) \{}
  
\NormalTok{  are_almost_same_ <-}\StringTok{ }\ControlFlowTok{function}\NormalTok{(string1, string2) \{}
  
\NormalTok{    chars1 <-}\StringTok{ }\KeywordTok{str_split}\NormalTok{(string1, }\StringTok{""}\NormalTok{)[[}\DecValTok{1}\NormalTok{]]}
\NormalTok{    chars2 <-}\StringTok{ }\KeywordTok{str_split}\NormalTok{(string2, }\StringTok{""}\NormalTok{)[[}\DecValTok{1}\NormalTok{]]}
    
    \KeywordTok{sum}\NormalTok{(chars1 }\OperatorTok{==}\StringTok{ }\NormalTok{chars2) }\OperatorTok{==}\StringTok{ }\DecValTok{25}
\NormalTok{  \}}
  
  \KeywordTok{map2_lgl}\NormalTok{(vector1, vector2, are_almost_same_)}
\NormalTok{\}}

\NormalTok{get_common_letters_ <-}\StringTok{ }\ControlFlowTok{function}\NormalTok{(string1, string2) \{}
  
\NormalTok{  chars1 <-}\StringTok{ }\KeywordTok{str_split}\NormalTok{(string1, }\StringTok{""}\NormalTok{)[[}\DecValTok{1}\NormalTok{]]}
\NormalTok{  chars2 <-}\StringTok{ }\KeywordTok{str_split}\NormalTok{(string2, }\StringTok{""}\NormalTok{)[[}\DecValTok{1}\NormalTok{]]}
  
  \KeywordTok{paste0}\NormalTok{(chars1[chars1 }\OperatorTok{==}\StringTok{ }\NormalTok{chars2], }\DataTypeTok{collapse =} \StringTok{""}\NormalTok{)}
  
\NormalTok{\}}

\NormalTok{puzzle_input }\OperatorTok
\StringTok{  }\KeywordTok{list}\NormalTok{(}\DataTypeTok{x =}\NormalTok{ ., }\DataTypeTok{y =}\NormalTok{ .) }\OperatorTok
\StringTok{  }\KeywordTok{cross_df}\NormalTok{(}\DataTypeTok{.filter =} \StringTok{`}\DataTypeTok{==}\StringTok{`}\NormalTok{) }\OperatorTok
\StringTok{  }\KeywordTok{mutate}\NormalTok{(}\DataTypeTok{are_almost_same =} \KeywordTok{are_almost_same}\NormalTok{(x, y)) }\OperatorTok\StringTok{ }
\StringTok{  }\KeywordTok{filter}\NormalTok{(are_almost_same) }\OperatorTok\StringTok{ }
\StringTok{  }\KeywordTok{slice}\NormalTok{(}\DecValTok{1}\NormalTok{) }\OperatorTok\StringTok{ }\CommentTok{# because of the cross_df()}
\StringTok{  }\NormalTok{\{}\KeywordTok{get_common_letters_}\NormalTok{(.}\OperatorTok{$}\NormalTok{x, .}\OperatorTok{$}\NormalTok{y)\}}
\end{Highlighting}
\end{Shaded}

\begin{verbatim}
## [1] "ivjhcadokeltwgsfsmqwrbnuy"
\end{verbatim}

\chapter{Methods}\label{methods}

We describe our methods in this chapter.

\chapter{Applications}\label{applications}

Some \emph{significant} applications are demonstrated in this chapter.

\section{Example one}\label{example-one}

\section{Example two}\label{example-two}

\chapter{Final Words}\label{final-words}

We have finished a nice book.

\bibliography{book.bib,packages.bib}


\end{document}
